
\section{Introdução} 


%------------------------------------------------

\begin{frame}
    \frametitle{O fenômeno de convecção atmosférica}

    Segundo Charles A. Doswell III, meteorologista americano, o fenômeno de convecção atmosférica, pode ser definido como:

    \vspace{1em} % Adiciona um espaço vertical de 1em

    \begin{quote}
        \itshape
        ``De um modo geral, a convecção refere-se ao transporte de uma determinada propriedade através do movimento de um fluido, na maioria das vezes com referência ao transporte de calor''
    \end{quote}

    \begin{flushright}
        \tiny
        \textit{Fonte: Doswell III, C. A., \textbf{Severe Convective Storms}, American Meteorological Society, 1996.}
    \end{flushright}
\end{frame}
%------------------------------------------------

\begin{frame}
    \frametitle{\textit{Finite Amplitude Free Convection as an Initial Value Problem—I}}
    
    Em 1962, Barry Saltzman publica o artigo que intitula este slide. Nele, Saltzman realiza experimentos meteorológicos e hidrodinâmicos. O artigo, tem dois objetivos:
    \vspace{0.5em}
    \begin{enumerate}
        \item Formular um modelo matemático para fenômenos de convecção de natureza não-linear;
        
        \item Determinar um método de solução de um caso de movimento convectivo dependente do tempo bidimensionais.
    \end{enumerate}
\end{frame}

%------------------------------------------------

\begin{frame}[t]
    \frametitle{\small Equação desenvolvida por Saltzman}
    
    {\small
    \begin{equation}
        \frac{a}{(1 + a^2)^k} \psi = x(t)\sqrt{2} \sin \left(\frac{\pi u}{a}\right) \sin \left(\frac{\pi v}{H}\right)
    \end{equation}
    \begin{equation}
        \frac{\pi R_o \theta}{R \Delta T} = y(t)\sqrt{2} \cos \left(\frac{\pi u}{a}\right) \sin \left(\frac{\pi v}{H}\right) - z(t) \sin \left(\frac{2\pi}{H}v\right)
    \end{equation}
    }

    {\scriptsize
    Onde:
    \begin{itemize}
        \item $u$: coordenada espacial horizontal ($m$);
        \item $x(t), y(t), z(t)$: coeficientes dependentes do tempo (amplitudes) (Unidade depende do contexto);
        \item $\frac{\pi}{H}$: inverso da profundidade da camada de fluido (máximo de $v$) ($m^{-1}$);
        \item $a$: parâmetro de geometria;
        \item $Ra$: número de Rayleigh;
        \item $R_c$: valor crítico de $Ra$ ($R_c = \pi^4(1 + a^2)^3/a^2$);
        \item $\Delta T$: diferença de temperatura total ($K$).
    \end{itemize}
    }
\end{frame}
%------------------------------------------------

\begin{frame}[t]
    \frametitle{Equação desenvolvida por Lorenz}
    \scriptsize
    Lorenz simplificou a equação desenvolvida por Saltzman, eliminou as funções trigonométricas e adotou equações diferenciais ordinárias. Temos o seguinte resultado:

    \begin{align*}
        \begin{cases}
            \dfrac{dx}{dt} &= \sigma(y-x) \\[6pt]
            \dfrac{dy}{dt} &= x(\rho - z) - y \\[6pt]
            \dfrac{dz}{dt} &= xy - \beta z
        \end{cases}
    \end{align*}

    \vspace{0.2cm}

    \begin{itemize}
        \item $\sigma$ controla a sensibilidade do sistema à diferença entre as variáveis $x$ e $y$ (número de Prandtl). 
        \item $\rho$ está associado à taxa de convecção do sistema (número de Rayleigh).
        \item $\beta$ está associado à geometria do sistema e à diferença entre as taxas de crescimento das variáveis $x$ e $z$.
    \end{itemize}
\end{frame}
    
%------------------------------------------------